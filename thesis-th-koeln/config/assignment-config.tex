\documentclass[a4paper, 11pt,oneside]{article}

% ============================================  USERSETTINGS  ==========================================

\newcommand{\myTitle}{Small \LaTeX{} Assignment Template\xspace}
\newcommand{\mySubtitle}{Subtitle\xspace}
\newcommand{\myDegree}{\xspace}
\newcommand{\myDegreeCourse}{Digital Sciences\xspace}
\newcommand{\myName}{First Name Last Name }
\newcommand{\myProf}{Prof. Name\xspace}
\newcommand{\myProfTwo}{Prof. Name\xspace}
\newcommand{\myOtherProf}{Prof. Name\xspace}
\newcommand{\mySupervisor}{Supervisor Name\xspace}
\newcommand{\myFaculty}{Fakultät für Informatik und Ingenieurwissenschaften\xspace}
\newcommand{\myDepartment}{Department\xspace}
\newcommand{\myUni}{Technischen Hochschule Köln\xspace}
\newcommand{\myLocation}{Location, Country\xspace}
\newcommand{\myTime}{\today\xspace}
\newcommand{\myVersion}{version 1.0\xspace}
\newcommand{\myModule}{Research proposal}
\newcommand{\myTerm}{Winterterm 2021/2022}
\newcommand{\myMail}{test@test.com \xspace}
\newcommand{\myLogo}{assets/logo.pdf}
\newcommand{\myMatriculationNo}{1337 \xspace}



% === USERSETTINGS FORMATTING: TITLE ===

\renewcommand{\maketitle}{
	\begin{titlepage}
	  
    % logo
    \vspace{-15mm}
       	 \includegraphics[height=30mm]{\myLogo}
    \vspace{15mm}
    
    % title
    \begin{center}
      \LARGE\textsc\myModule\\[4ex]
      \Huge \textbf{\myTitle}\\[2ex]
      \hrule\par\rule{0pt}{4ex}
      \large vorgelegt an der \myUni\\
      \large im Studiengang \myDegreeCourse\\
      \large der \myFaculty\\
      \normalsize\date
      \vfill
    \end{center}

    \vfill
    % info
    \begin{center}
      \begin{minipage}{0.49\textwidth}
        \large
        \begin{tabular}[t]{l|l}
        & \textbf{Von} \myName \\[2pt]
        & \textbf{Mail} \hrefblue{mailto:test@test.com}{test@test.com} \\[2pt]
        & \textbf{Matrikelnr.} \myMatriculationNo \\[2pt]
        & \textbf{Erstprüfer} \myProf\\[2pt]
        & \textbf{Zweitprüfer} \myProfTwo\\[2pt]
        \end{tabular}
    \end{minipage}
    \end{center}
    
    \vfill
    \centering \today

  \end{titlepage}
}





% ============================================  PACKAGES  ==============================================

% === TEMPLATE RELATED: Remove in case of usage! ===
\usepackage{lipsum} % Package to generate dummy text throughout this template

% === %unsorted% ===
\usepackage{filecontents}
\usepackage{lscape}


% === FONT & LANGUAGE ===
% \usepackage[ngerman]{babel} % set language
\usepackage[main=ngerman,english]{babel} % set multi-language document

\usepackage{microtype} % Slightly tweak font spacing for aesthetics
\usepackage[utf8]{inputenc}
\usepackage[T1]{fontenc} % needed if Umlauts are used
\usepackage[utf8]{inputenc}
\usepackage{lmodern}
\usepackage{xspace} % adds a space unless the macro is followed by certain punctuation characters

% === COLORING ===
\usepackage[dvipsnames]{xcolor}

% === MARGINS ===
\usepackage[a4paper]{geometry}
\geometry{left=3cm,right=2cm,top=1.5cm,bottom=1cm,headheight=14.599pt,headsep=1cm,textheight=245mm,textwidth=160mm,includeheadfoot,footskip=1cm}%Seitenränder
\usepackage[onehalfspacing]{setspace}%Zeilenabstand

\renewcommand{\\}{\vspace*{0.5\baselineskip} \newline}

% === MULTICOLUMN ===
\usepackage{multicol} 

% === CITATION ===
\usepackage[
  backend=biber,
  citestyle=ieee
]{biblatex}
\addbibresource{bibliography.bib} %Imports 

% === Abstract ===
\usepackage{abstract} % Allows abstract customization
\renewcommand{\abstractnamefont}{\normalfont\bfseries} % Set the "Abstract" text to bold
\renewcommand{\abstracttextfont}{\normalfont\small\itshape} % Set the abstract itself to small italic text

% === SECTIONS ===
%\renewcommand\thesection{\Roman{section}} % sections with roman numerals
%\renewcommand\thesubsection{\Roman{subsection}} % subsections with roman numerals

%\titleformat{\section}[block]{\large\scshape\centering}{\thesection.}{1em}{} % new style of section titles
%\titleformat{\subsection}[block]{\large}{\thesubsection.}{1em}{} % new style of section titles

% === HEADERS & FOOTERS
\usepackage{fancyhdr} 
\pagestyle{fancy} % set for all pages
\fancyhead{} % Blank out the default header
\fancyfoot[C]{\thepage} % Number of page centered in footer
\fancyhead[L]{\nouppercase{\rightmark}} % Custom header text
\fancyhead[C]{} % Custom header text
\fancyhead[R]{} % Custom header text
\renewcommand{\headrulewidth}{0.5pt}
\renewcommand{\headrule}{\hbox to\headwidth{%
  \color{Gray} \leaders\hrule height \headrulewidth\hfill}} % thickness of lines
\renewcommand{\sectionmark}[1]{ \markright{#1}{} }

% === FIGURES ===
\usepackage{graphicx}
\usepackage{caption}

% === TABLES ===
\usepackage{booktabs}  % Horizontal rules in tables
\usepackage{array} % extends tabular

% === ROTATION ===
\usepackage{rotating}

% === Algorithms ===
\usepackage{algorithm}
\usepackage{algorithmic}

% === Hyperlinks ===
\usepackage[hidelinks,breaklinks,pdfpagelabels,pdfstartview=FitH,bookmarksopen=true,bookmarksnumbered=true,plainpages=false,hypertexnames=false]{hyperref}
\newcommand{\hrefblue}[3][MidnightBlue]{\href{#2}{\color{#1}{#3}}}

% === FIXES ===
\newcommand{\theHalgorithm}{\arabic{algorithm}} % hyperref and algorithmic misbehave sometimes
\usepackage{float} % Required for tables and figures in the multi-column environment - they need to be placed in specific locations with the [H] (e.g. \begin{table}[H])
\usepackage{paralist} % less space between bullet points

% ============================================  FINETUNING  =============================================

%\renewcommand*{\backref}[1]{}
%\renewcommand*{\backrefalt}[4]{{\footnotesize [%
%    \ifcase #1 Not cited.%
%    \or Cited on page~#2%
%    \else Cited on pages #2%
%    \fi%
%]}
